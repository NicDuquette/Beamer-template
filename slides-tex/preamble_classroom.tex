% This preamble defines properties we only want to apply to the in-class, on-a-projector
% versions of the slides. A sister file, preamble_handout.tex, defines all of these
% differently for student handouts, which changes the handling of the same macros
% across slide themes. A third file, preamble_general.tex, defines Latex macros 
% that are the same for both slide formats.


							%%%%%%%%%%%%%%%%%%%%%%%%
							%  	  DEFINE COLORS    %
							%%%%%%%%%%%%%%%%%%%%%%%%


    % SET BEAMER COLORS
    % DEFAULT USES WHITE, LIGHT BLUE
    % TEXT ON BLACK BACKGROUND
    % http://cloford.com/resources/colours/500col.htm											
    % Dark == true => use dark theme
    	\definecolor{lightslateblue}{RGB}{192,192,255}			% Define a blue that stands out against black [old version: 132,112,255]
    	\setbeamercolor{background canvas}{bg=black}			% background - black
    	\setbeamercolor{normal text}{fg=white}					% white text
    	\setbeamercolor{block title}{bg=black,fg=white}			% bg=background, fg= foreground
    	\setbeamercolor{titlelike}{fg=lightslateblue}			% Slide title color
    	\setbeamercolor{title}{fg=lightslateblue}				%  Front slideTitle color
    	\setbeamercolor{enumerate title}{fg=lightslateblue}		%
    	\setbeamercolor{enumerate item}{fg=lightslateblue}		%
    	\setbeamercolor{enumerate subitem}{fg=lightslateblue}	%
    	\setbeamercolor{enumerate subsubitem}{fg=lightslateblue}	%
    	\setbeamercolor{itemize title}{fg=lightslateblue}		%
    	\setbeamercolor{itemize item}{fg=lightslateblue}		% Bullet item
    	\setbeamercolor{itemize subitem}{fg=lightslateblue}		% Bullet sub-item
    	\setbeamercolor{itemize subsubitem}{fg=lightslateblue}	% Bullet sub-sub-item
    	\setbeamercolor{section in toc}{fg=lightslateblue}
	
	
%	120 120 255
    	\definecolor{xblue}{RGB}{100,150,255}
        \definecolor{xred}{RGB}{255,81,81}
        \definecolor{xyellow}{RGB}{240,228,66}
        \definecolor{xgreen}{RGB}{0,250,95}
        \definecolor{xorange}{RGB}{255, 140, 33}
        \definecolor{xbrown}{RGB}{144, 67, 19}
        \definecolor{xpurple}{RGB}{232, 179, 252}
        
        
        
 	\definecolor{xmono}{RGB}{255,255,255}	% "mono" is white if background is black,
	\definecolor{xbg}{RGB}{0,0,0}			% black if it is white; xbg is opposite	
	\definecolor{xgrey}{RGB}{128,128,128}			% grey											
		
     
							%%%%%%%%%%%%%%%%%%%%%%%%
							%  DEFINE CODE BLOCKS  %
							%%%%%%%%%%%%%%%%%%%%%%%%

	% BOOLEAN \codeblockson will be passed from the command line in the final pass and will
	% trigger the full, syntax-highlighting version of the minted package. When we want to save
	% time, or when we're tweaking this in a place like TexShop where the --shell-escape
	% option minted demands is unavailable, we will use draft=true to make things faster
	% and uncolored.
\ifdefined\codeblockson
	\usepackage[final=true]{minted}
\else
	\usepackage[draft=true]{minted}
\fi

\usemintedstyle{mydark}

	% nicolasduquette@Prices-MacBook-Pro ~ % which pygmentize
	% /Library/Frameworks/Python.framework/Versions/3.8/bin/pygmentize
	% cd /Library/Frameworks/Python.framework/Versions/3.8/lib/python3.8/site-packages/pygments/styles

	% Create environments {pythoncode}, {statacode}, {latexcode}, {bashcode}
	% for code blocks on slides
\newminted{python}{fontsize=\scriptsize, 
                   gobble=4,
                   style=mydark
                   } 

\newminted{latex}{fontsize=\scriptsize, 
                   gobble=4,
                   style=mydark
                   }                    

\newminted{stata}{fontsize=\scriptsize, 
                   %linenos,
                   %numbersep=8pt,
                   gobble=0,
                   %frame=lines,
                   style=mydark, %Others: stata, stata-light, stata-dark
                   framesep=3mm
                   }  
                   
\newminted{bash}{fontsize=\scriptsize, 
                   %linenos,
                   %numbersep=8pt,
                   gobble=0,
                   %frame=lines,
                   style=mydark, %Others: stata, stata-light, stata-dark
                   framesep=3mm
                   }  
      
      % Create simple commands \stata{}, \python{} for inline highlights        
\newcommand{\stata}[1]{\mintinline{stata}{#1}}
\newcommand{\python}[1]{\mintinline{python}{#1}}
\newcommand{\latexinline}[1]{\mintinline{latex}{#1}}

	% Whole-slide code block
\newcommand{\InsertStataFrame}[2]{\begin{frame}\frametitle{#1} {\footnotesize \inputminted[style=mydark,bgcolor=xbg]{stata}{#2}} \end{frame}}
\newcommand{\InsertPythonFrame}[2]{\begin{frame}\frametitle{#1} {\footnotesize \inputminted[style=mydark,,bgcolor=xbg]{python}{#2}} \end{frame}}


        										
     
							%%%%%%%%%%%%%%%%%%%%%%%%%
							% DEFINE SPECIAL SLIDES %
							%%%%%%%%%%%%%%%%%%%%%%%%%
							
	% Add a transition frame that does not get added to the table of contents
\newcommand{\transitionframe}[1]{ 
   {
    \setbeamercolor{background canvas}{bg=lightslateblue}
    \begin{frame}
         \begin{center}
        { \Huge \textcolor{black}{#1}}
      \end{center}
    \end{frame}
    }
}

	% Add a section frame and a \section with transitionframe format
\newcommand{\sectionframe}[1]{
    \section{#1}
    {
    \setbeamercolor{background canvas}{bg=lightslateblue}
    \begin{frame}
         \begin{center}
        { \Huge \textcolor{black}{#1}}
      \end{center}
    \end{frame}
    }
}


	% Add a special section frame for the end
	% in the classroom but not the handouts
\newcommand{\theendframe}{
    {
    \setbeamercolor{background canvas}{bg=black}
    \begin{frame}
         \begin{center}
        { \Huge \textcolor{white}{End}}
      \end{center}
    \end{frame}
    }
}

	% Answers to in-class exercises (not provided in handouts)
\newenvironment{solutionframe}{
	\begin{frame}
}{
	\end{frame}
}
\newenvironment{solutionimageframe}{
	\begin{frame}[plain]
}{
	\end{frame}
}
     
							%%%%%%%%%%%%%%%%%%%%%%%%
							% DEFINE FILE PATHWAYS %
							%%%%%%%%%%%%%%%%%%%%%%%%

\newcommand{\figpath}{../figs_dark}				% Path to scheme-optimized images (for in-class projection)
\newcommand{\figboth}{../figs_both}				% Path to unchanging images
\newcommand{\tablepath}{../tables_tex}			% Path to tables in Latex format





