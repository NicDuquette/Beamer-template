  	%PGP colors
	\definecolor{xblue}{RGB}{0,114,178}			% #0072B2
	\definecolor{xdarkblue}{RGB}{0,82,146}		% #005292
    \definecolor{xred}{RGB}{255,30,30}			% #FF1E1E
    \definecolor{xyellow}{RGB}{240,228,66}
    \definecolor{xgreen}{RGB}{0,158,85}			% #009E55
    \definecolor{xorange}{RGB}{213, 111, 62}		% #D56F3E
    \definecolor{xbrown}{RGB}{76, 35, 10}    
    \definecolor{xpurple}{RGB}{163, 82, 143}		% #A3528F
    
    
 	\definecolor{xmono}{RGB}{0,0,0}		% "mono" is white if background is black,
	\definecolor{xbg}{RGB}{255,255,255}	% black if it is white; xbg is opposite
 

		% Add a transition frame that does not get added to the TOC
	\newcommand{\transitionframe}[1]{
	{
        \setbeamercolor{background canvas}{bg=xyellow}
        \begin{frame}
             \begin{center}
            { \Huge \textcolor{black}{#1}}
          \end{center}
        \end{frame}
     }
    }



    	% Add a section frame and a \section with special format
    \newcommand{\sectionframe}[1]{
        \section{#1}
        {
        \setbeamercolor{background canvas}{bg=xyellow}
        \begin{frame}
             \begin{center}
            { \Huge \textcolor{black}{#1}}
          \end{center}
        \end{frame}
        }
    }
    
    
	% Add a special section frame for the end
	% in the classroom but not the handouts
\newcommand{\theendframe}{
}

    
    	% Answers to in-class exercises (not provided in handouts)
\newenvironment{solutionframe}{
	\begin{frame}<1-| handout:0>[noframenumbering]
}{
	\end{frame}
}
\newenvironment{solutionimageframe}{
	\begin{frame}<1-| handout:0>[noframenumbering]
}{
	\end{frame}
}
     
    
    
    	% File path to handout versions of tables and charts
\newcommand{\figpath}{../figs_light}				% Path to figures
\newcommand{\figboth}{../figs_both}				% Path to unchanging images
\newcommand{\tablepath}{../tables_tex}			% Path to tables generated from Stata
\newcommand{\croppath}{../floats_snipped}		% Path to images of tables and figs from cited papers
    
    

     
							%%%%%%%%%%%%%%%%%%%%%%%%
							%  DEFINE CODE BLOCKS  %
							%%%%%%%%%%%%%%%%%%%%%%%%

\ifdefined\codeblockson
	\usepackage[final=true]{minted}
\else
	\usepackage[draft=true]{minted}
\fi

	%https://pygments.org/demo/#try
\newminted{python}{fontsize=\scriptsize, 
                   gobble=4,
                   style=mylight
                   } 
                   
\newminted{latex}{fontsize=\scriptsize, 
                   gobble=4,
                   style=mylight
                   }                       

\newminted{stata}{fontsize=\scriptsize, 
                   %linenos,
                   %numbersep=8pt,
                   gobble=0,
                   %frame=lines,
                   style=mylight, %Others: stata, stata-light, stata-dark
                   framesep=3mm
                   }   
                   
\newminted{bash}{fontsize=\scriptsize, 
                   %linenos,
                   %numbersep=8pt,
                   gobble=0,
                   %frame=lines,
                   style=mylight, %Others: stata, stata-light, stata-dark
                   framesep=3mm
                   }        

\newcommand{\stata}[1]{\mintinline{stata}{#1}}
\newcommand{\latexinline}[1]{\mintinline{latex}{#1}}
\newcommand{\python}[1]{\mintinline{python}{#1}}


	% Whole-slide code block
\newcommand{\InsertStataFrame}[2]{\begin{frame}\frametitle{#1} {\footnotesize \inputminted[style=mylight]{stata}{#2}} \end{frame}}
\newcommand{\InsertPythonFrame}[2]{\begin{frame}\frametitle{#1} {\footnotesize \inputminted[style=mylight]{python}{#2}} \end{frame}}



	% TEXT CODE COMMANDS
	\newcommand{\CMD}[1]{\texttt{\textcolor{xgreen}{#1}}}
	\newcommand{\VBL}[1]{\texttt{\textcolor{xred}{#1}}}	
	\newcommand{\CODE}[1]{\texttt{\textcolor{xblue}{#1}}}


 
 	% VERBATIM CODE COMMANDS
%	% http://scott.sherrillmix.com/blog/programmer/displaying-code-in-latex/
	\newcommand\Cemph[1]{\textcolor[rgb]{1,0,0}{\textbf{#1}}}		% Emphasis
	\newcommand\Command[1]{\textcolor{xgreen}{{#1}}}				% Command
	\newcommand\Var[1]{\textcolor{xred}{{#1}}}				% Variables
	\newcommand\Comment[1]{\textcolor[rgb]{0.35,0.35,0.35}{{#1}}}				% Comments


