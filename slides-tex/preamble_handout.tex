% This preamble defines properties we only want to apply to slides given to students
% as a handout. A sister file, preamble_classroom.tex, defines all of these
% differently for projection during lecture, which changes the handling of the same macros
% across slide themes. A third file, preamble_general.tex, defines Latex macros
% that are the same for both slide formats.
    
    

                           %%%%%%%%%%%%%%%%%%%%%%%%
                           % DEFINE FILE PATHWAYS %
                           %%%%%%%%%%%%%%%%%%%%%%%%

\newcommand{\figpath}{../figs_light}				% Path to figures optimized for print
\newcommand{\figboth}{../figs_both}				% Path to unchanging images
\newcommand{\tablepath}{../tables_tex}			% Path to tex-formatted tables
    

							%%%%%%%%%%%%%%%%%%%%%%%%
							%  	  DEFINE COLORS    %
							%%%%%%%%%%%%%%%%%%%%%%%%    


	% Define special x-colors that can be used for consistent appearance
	% within themes
\definecolor{xblue}{RGB}{0,114,178}			% #0072B2
\definecolor{xdarkblue}{RGB}{0,82,146}		% #005292
\definecolor{xred}{RGB}{255,30,30}			% #FF1E1E
\definecolor{xyellow}{RGB}{240,228,66}
\definecolor{xgreen}{RGB}{0,158,85}			% #009E55
\definecolor{xorange}{RGB}{213, 111, 62}		% #D56F3E
\definecolor{xbrown}{RGB}{76, 35, 10}    
\definecolor{xpurple}{RGB}{163, 82, 143}		% #A3528F


\definecolor{xmono}{RGB}{0,0,0}		% "mono" is white if background is black,
\definecolor{xbg}{RGB}{255,255,255}	% black if it is white; xbg is opposite




     
							%%%%%%%%%%%%%%%%%%%%%%%%
							%  DEFINE CODE BLOCKS  %
							%%%%%%%%%%%%%%%%%%%%%%%%

	% BOOLEAN \codeblockson will be passed from the command line in the final pass and will
	% trigger the full, syntax-highlighting version of the minted package. When we want to save
	% time, or when we're tweaking this in a place like TexShop where the --shell-escape
	% option minted demands is unavailable, we will use draft=true to make things faster
	% and uncolored.
\ifdefined\codeblockson
	\usepackage[final=true]{minted}
\else
	\usepackage[draft=true]{minted}
\fi

\usemintedstyle{mylight}
%\usemintedstyle{autumn}

\newminted{python}{fontsize=\scriptsize, 
                   gobble=4,
                   style=mylight
%                   style=autumn
                   } 
\newminted{stata}{fontsize=\scriptsize, 
                   gobble=0,
                   style=mylight, 
%                   style=autumn, 
                   framesep=3mm
                   }   
\newcommand{\stata}[1]{\mintinline{stata}{#1}}
\newcommand{\python}[1]{\mintinline{python}{#1}}


	% Whole-slide code block
\newcommand{\InsertStataFrame}[3][\footnotesize]{\begin{frame}\frametitle{#2} {#1 \inputminted[style=mylight,obeytabs=true,tabsize=4]{stata}{#3}}  \end{frame}}

\newcommand{\InsertPythonFrame}[3][\footnotesize]{\begin{frame}\frametitle{#2} {#1 \inputminted[style=mylight,obeytabs=true,tabsize=4]{python}{#3}} \end{frame}}
%
%\newcommand{\InsertStataFrame}[3][\footnotesize]{\begin{frame}\frametitle{#2} {#1 \inputminted[style=autumn,obeytabs=true,tabsize=4]{stata}{#3}}  \end{frame}}
%
%\newcommand{\InsertPythonFrame}[3][\footnotesize]{\begin{frame}\frametitle{#2} {#1 \inputminted[style=autumn,obeytabs=true,tabsize=4]{python}{#3}} \end{frame}}





							%%%%%%%%%%%%%%%%%%%%%%%%%
							% DEFINE SPECIAL SLIDES %
							%%%%%%%%%%%%%%%%%%%%%%%%%
 

		% Add a transition frame that does not get added to the TOC
	\newcommand{\transitionframe}[1]{
	{
        \setbeamercolor{background canvas}{bg=xyellow}
        \begin{frame}
             \begin{center}
            { \Huge \textcolor{black}{#1}}
          \end{center}
        \end{frame}
     }
    }



    	% Add a section frame and a \section with special format
    \newcommand{\sectionframe}[1]{
        \section{#1}
        {
        \setbeamercolor{background canvas}{bg=xyellow}
        \begin{frame}
             \begin{center}
            { \Huge \textcolor{black}{#1}}
          \end{center}
        \end{frame}
        }
    }
    
    
	% Add a special section frame for the end
	% in the classroom but not the handouts
\newcommand{\theendframe}{
}

    
    	% Answers to in-class exercises (not provided in handouts)
\newenvironment{solutionframe}{
	\begin{frame}<1-| handout:0>[noframenumbering]
}{
	\end{frame}
}

     

